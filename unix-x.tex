%%%%%%%%%%%%%%%%%%%%%%%%%%%%%%%%%%%%%%%%%%%%%%%%%%%%%%%%%%%%%%%%%%%%%
%
% Written by Per Foreby <per.foreby@kansli.lth.se> 1995-2012
% Updated by Hannes Åström <dat15has@student.lu.se> 2018
% Updated by Nils Ceberg <ni7228ce-s@student.lu.se> 2019
%
% Permission is granted to non-profit organisations to use this
% code any way you want, as long as you don't pretend you wrote it!
%
% Credit where credit is due!
%
%%%%%%%%%%%%%%%%%%%%%%%%%%%%%%%%%%%%%%%%%%%%%%%%%%%%%%%%%%%%%%%%%%%%%%

\documentclass[a4paper,twocolumn]{book}
\usepackage[T1]{fontenc}
\usepackage[utf8]{inputenc}
\usepackage[swedish]{babel}

%\usepackage[draft,nogray]{unix-x}		% Preview
%\usepackage[draft]{unix-x}			% Draft printing
%\usepackage[html]{unix-x}			% For latex2html
\usepackage{unix-x}				% Final printing

% Detta ligger efter \usepackage{unix-x} eftersom \ifpdf definierad där.
\ifpdf
  \usepackage[pdftex]{graphics}
  \pdfcompresslevel=9
%  \usepackage[pdftex,hyperindex,colorlinks]{hyperref}
  \usepackage[pdftex,hyperindex,hidelinks]{hyperref}
\else
 %\usepackage[draft]{graphics}
  \usepackage[final]{graphics}
  \usepackage[dvips,hidelinks]{hyperref}
\fi

\usepackage{tabularx}

\usepackage{makeidx}
\makeindex

\begin{document}

%%%%%%%%%%%%%%%%%%%%%%%%%%%%%%%%%%%%%%%%%%%%%%%%%%%%%%%%%%%%%%%%%%%%%%%%%%%%%%
\frontmatter
%%%%%%%%%%%%%%%%%%%%%%%%%%%%%%%%%%%%%%%%%%%%%%%%%%%%%%%%%%%%%%%%%%%%%%%%%%%%%%

\begin{titlepage}
  \vspace*{30mm}
  \begin{center}
    \sffamily
    \renewcommand{\baselinestretch}{1.1}
    \Huge\bfseries Introduktion till \\[2mm]
      LTH:s Linuxdatorer \\[7mm]
    \large Lunds universitet \\[5mm]
    Version 1.80 \\
    16 juni 2018
  \end{center}
\end{titlepage}

\onecolumn
\thispagestyle{empty}
\hbox{}\vfill\noindent
\twocolumn

{\parskip 0pt%
  \tableofcontents}             % Run toc tighter.

%=============================================================================
\chapter{Förord}
%=============================================================================
\label{cha:forord}

Den här handledningen är en kort introduktion till Linuxbaserade operativsystem och 
hur man kan använda dem.

\section*{Handledningens uppläggning}

Denna handledning består av fyra kapitel. Kapitlen \emph{Grunderna} och \emph{Påbyggnad} ska ungefär
motsvara övningspassen på datorstugan. Kapitlet \emph{Överkurs}
behandlar saker som inte är nödvändiga att lära sig den första veckan,
men som kan vara bra att kunna för att utnyttja datorerna effektivt under
utbildningen. Kapitlet \emph{Hemifrån} innehåller en del tips om hur man kan
komma åt skolans datorer utifrån.

Handledningen innehåller rätt mycket information, och har man inte använt Linux
 tidigare så är det mycket att sätta sig in i.
Det är därför lämpligt att fortsätta på egen hand med det som ni
inte hinner med under datorstugan.

\section*{Övningsexempel}

På sida \pageref{cha:uppgifter} finns det en del övningsuppgifter.
Det kan vara en bra idé att först titta igenom uppgifterna för att sedan
gå igenom häftets innehåll och successivt lösa övningsuppgifterna.

Det finns dock inte övningar på
allting som handledningen omfattar, så läsaren uppmanas att själv
prova kommandon allt eftersom de introduceras i texten.

För de övningar där det är tillämpligt finns svar i
appendix~\ref{answers}.

\section*{Simulerad text}

När handledningen visar exempel på in- och utmatning i ett
terminalfönster ser det ut så här:
\begin{example}
lo-1\{e01nn\}: \textbf{ls}              
a.out        uppg2        uppg3.pas~
uppg1        uppg2.pas
uppg1.pas    uppg3.pas
lo-1\{e01nn\}: \textbf{echo "hej hopp i lingonskogen"}
hej hopp i lingonskogen
\end{example}
\emph{lo-1} är datorns namn och \emph{e01nn} är den inloggade användaren. Den text som användaren själv skriver in markeras med fetstil (här kommandot "ls" för att läsa en katalog och echo för att skriva ut text). Övrig text är sådant som datorn skriver till skärmen.

\section*{Tekniska detaljer}

Denna handledning är producerad med typsättningsprogrammet \LaTeX. De
teckensnitt som har använts är Times, Avant Garde och Courier.
Skärmbilderna är dumpade från en av våra datorer med programmet
\emph{import} och sedan redigerade med bildbehandlingsprogrammet GIMP.

\subsection*{Kommentarer välkomna}

Har ni synpunkter på hur handledningen kan bli bättre, så tar vi
gärna emot dem, helst via e-post till \ST{datateknik@kansli.lth.se} eller till
\ST{infocom@kansli.lth.se}.

%%%%%%%%%%%%%%%%%%%%%%%%%%%%%%%%%%%%%%%%%%%%%%%%%%%%%%%%%%%%%%%%%%%%%%%%%%%%%%
\mainmatter
%%%%%%%%%%%%%%%%%%%%%%%%%%%%%%%%%%%%%%%%%%%%%%%%%%%%%%%%%%%%%%%%%%%%%%%%%%%%%%

%=============================================================================
\chapter{Grunderna}
%=============================================================================
\label{cha:grunderna}



\section{Inledning}


\subsection{Datorerna}
\index{datorerna}

Linuxdatorerna som används på LTH är vanliga PC-maskiner av olika
modeller och åldrar. Vi har 13 olika Linuxrum med
vardera 10--15 arbetsstationer:
\begin{description}
\item[Venus (E:0422)] 
\item[Mars (E:0421)]
\item[Jupiter (E:0420)] \mbox{}\\
  Rummen finns i norra delen av E-husets källare. 
\item[Hacke (E:0153)]
\item[Panter (E:0154-E:0155)] 
\item[Lo (E:0210-0211)]
\item[Val (E:0507-E:0508)]
\item[Falk (E:0509-e:0510)]
\item[Varg (E:0518-E:0519)] \mbox{}\\
  Rum i södra delen av E-husets källare.
\item[Alfa (E:2119)] 
\item[Beta (E:2118)] 
\item[Gamma (E:2121)] \mbox{}\\
  På andra våningen i E-huset.
\item[Fys:H321] \mbox{}\\
  På Fysicum.
\end{description}

I E-huset finns även ett antal rum med datorer som kör Windows, och i
Matematikhuset finns också ett Windowsrum.
Windowsdatorerna behandlas inte alls i denna handledning.

LTH har även studentdatorer i Studiecentrum, M-huset, V-huset, A-huset, IKDC,
Kemicentrum och Fysicum. Med undantag för sal H321 på Fysicum
så kör samtliga av dessa datorer Windows. Det är alltså bara i E-huset och
på Fysicum som det finns Linuxdatorer.

Som student på LTH kan man använda datorer i alla husen, men programvarorna i
respektive hus är anpassade till undervisningen i huset. De övriga husen
saknar alltså en del av de program som ni kan behöva, och på motsvarande sätt
finns till exempel inte M-husets CAD-program i E-huset.

Datorerna i LTH:s studiecentrum har inga husspecifika programvaror, utan
endast vår grundinstallation. De är i första hand tänkta för enklare
uppgifter, som till exempel rapportskrivning.

Sedan finns det några datorrum som tillhör institutioner, men som är kopplade
till vårt gemensamma datorsystem för att underlätta för studenterna. Sådana
datorer hittar man bland annat på Fysicum, reglerteknik och elektro- och
informationsteknik (EIT).

% TODO complete rewrite; reread
\subsection{Linuxbaserade operativsystem}
\index{Unix}\index{operativsystem}

Ett operativsystem är en uppsättning program som administrerar datorns
interna arbete och erbjuder användaren ett antal färdiga kommandon. En central
del av ett operativsystem är kärnan (på engelska \emph{kernel}), som hanterar
de mest grundläggande uppgifterna såsom exempelvis kommunikationen med hårdvaran
och schemaläggning av processer.

Våra datorer använder operativsystemet Ubuntu, som är baserat på
operativsystemkärnan Linux. Linuxbaserade operativsystem tillhör vad som brukar
kallas de Unix-liknande operativsystemen, då de är baserade på samma konventioner
och fungerar på i princip samma sätt som det äldre operativsystemet Unix.

Unix började utvecklas på Bell Laboratories i USA 1969, och är idag ett
mycket moget och väl fungerande operativsystem. Unix och Unix-liknande system finns
i ett antal olika varianter; Linux är till exempel en helt fri variant av
ett Unix-liknande system vars källkod är öppen (det vill säga fritt tillgänglig).
Sett ur den vanliga användarens synvinkel är
dock de olika Unix-dialekterna mycket lika. Det är oftast bara avancerade
användare som märker skillnaderna. \textbf{Trots att vi i den här handledningen talar
om Linux är den alltså även fullt applicerbar på andra Unix-liknande system.}

Andra mer eller mindre kända Unix-liknande system är MacOS / OSX, diverse BSD-varianter,
Solaris, med flera.

\subsection{Linux och terminalen}
\index{Unix}\index{operativsystem}

I denna introduktion så kommer ni att bekanta er med Linux och dess terminal.
Terminalen är ett textbaserat program som visar alla sina utskrifter i text
och man styr den genom att skriva in textkommandon. 
Terminalen är ett naturligt redskap för programmerare, då det finns många program
som inte har något grafiskt användargränssnitt men som kan köras i terminalen. Detta ger programmeraren en mycket större verktygslåda än en vanlig datoranvändare. 
Det kan kännas svårt, komplicerat och skrämmande att arbeta med en terminal ifall man inte har använt den förut, men man lär sig efterhand. Denna handledning är ger en stabil grund för vidare lärande.

Linux är utvecklat på ett sätt som gör att terminalen är väldigt kraftfull.
Man kan göra i stort sett allt via terminalen. Vissa saker kan vara enklare 
eller bekvämare att göra med ett grafiskt användargränssnitt, men det finns ändå
många fördelar med terminalen, till exempel så kan man i ett och samma kommando skicka data
mellan olika program om man vill behandla datan på flera sätt. Det går snabbt att
göra saker och om man utvecklar ett program så behöver man inte skriva ett 
användargränssnitt för att testa programmet.

Ni kommer under er utbildning läsa kurser där man skriver avancerade program som
inte kommer att ha ett användargränssnitt och det är då bra att ha bekantat sig
med hur terminalen fungerar.

\subsection{Fönstersystemet X}
\index{X Window System}
\index{fönstersystem}

De program som gör att man kan arbeta med olika fönster på den
grafiska bildskärmen kallas för ett \emph{fönstersystem}. Det system
som vi använder heter \emph{X Window System, Version 11},
eller kort och gott \emindex{X}. Det förekommer även andra
förkortningar som \emindex{X11} eller \emindex{X.org} (efter organisationen
som utvecklar systemet). Ibland används
även den felaktiga benämningen \emph{X Windows}.

Det program som ger fönstersystemet sitt karaktäristiska utseende
kallas för \emindex{fönsterhanterare} eller \emph{window manager}. Det
ansvarar bland annat för att rita ramar runt fönster och göra menyer, medan
varje program ansvara för resten av utseendet.

I dag går man ofta ett steg längre och använder något som kallas
\emindex{skrivbordsmiljö} (\emph{desktop environment}). Med en sådan så får
alla program ett gemensamt utseende och kan även kommunicera med varandra
bättre än om man bara har en fönsterhanterare.

\section{Datorstugan}

Ni som går igenom detta kompendium på introduktionsveckans datorstuga
har en övningsledare. Han/hon får betalt för att svara på frågor och
hjälpa dig genom övningarna. Tveka inte att utnyttja övningsledarens
kunskaper. 


\section{Inloggning}

För att kunna använda datorn måste man vara registrerad som användare
i datorsystemet. Det brukar kallas för att ha ett \emindex{konto}.

En användare identifieras med ett unikt \rmindex{användarnamn}
(\emindex{username}) och ett \rmindex{lösenord} (\emindex{password}).

Alla studenter har förhoppningsvis redan fått sitt konto hemskickat. Annars
kan man hämta ut det på studiecentrum (prata med din labbhandledare). Observera att det inte går att logga in
på datorerna förrän man har hämtat ut sitt LU-kort.


\section{Introduktion till Linux}

Om du skrev rätt användarnamn och lösenord slutförs inloggningen på några
sekunder.

På skärmen finns nu några \emindex{ikoner}, och längst till vänster  en så kallad
\emindex{panel}.


\subsection{Musoperationer} %%NOTE: Checka om man kan kopiera med scrollhjulet
\index{musoperationer}

Musen har tre knappar och fungerar ungefär som i Windows, men med en stor
skillnad: För att klippa och klistra behöver man bara markera texten med
vänsterknappen, sedan klistrar man in den genom att klicka på mittknappen
(rullhjulet). I de flesta program går det även att klippa och klistra med
tangentbordsoperationer, precis som i andra operativsstem.
Markera en bit text. Använd sedan \emph{Ctrl-C} för
att kopiera den eller \emph{Ctrl-X} för att klippa ut den. Flytta sedan
muspekaren dit du vill ha texten och tryck \emph{Ctrl-V} för att klistra in
texten.
\subsection{Val av språk}
Eftersom vi har en hel del gäststudenter på LTH så har vi valt att ha engelska
som standardspråk på datorerna. Vill man ha ett svenskt användargränssnitt så
väljer man det på inloggningsskärmen.

Om man väljer ett annat språk så ändras det i de flesta programmen. Det finns
dock en del program som inte är översatta.

\subsection{Skärmlåsning}
\index{Skärmlåsning}

I menyn kan man låsa skärmen manuellt. Den låser sig även efter
ett tag om man inte gör någonting. För att låsa upp igen rör man på musen och
får upp en ruta där man skriver in sitt lösenord. Det är inte tillåtet att
låsa skärmen under längre perioder. I så fall ska man i stället logga ut så
att någon annan kan använda datorn. Som tumregel kan man säga att 15
minuter kan vara ok, men det är absolut inte är tillåtet att reservera en
dator och sedan gå på en föreläsning i två timmar.

\section{Passerkort}

För att kunna komma in i datorrummen behöver man ett
\rmindex{passerkort} med en personlig kod. Passerkortet får man på studiecentrum.
Det fungerar även till ytterdörrarna, så att man kan komma i in i huset på
kvällar och helger. Kortet fungerar även som ID-kort, och på kvällar och
helger kan väktarna kräva att man visar upp sitt kort. 

\section{Lösenord}

Ditt användarnamn och \rmindex{lösenord} är nyckeln till ditt
datorkonto. Det är viktigt att du håller ditt lösenord hemligt så att
ingen annan kan komma åt din laborationsrapport eller göra något
otillåtet i ditt namn. \textbf{Avslöja aldrig ditt lösenord för någon.}

Ibland kan det förekomma mailutskick som säger att man ska skriva in sitt
lösenord på en websida för att få behålla sitt mailkonto.  Vi skickar
\textbf{aldrig} ut mail och ber studenterna skriva sina lösenord på en extern
websida. Det är bara ett sätt för spammare att komma över konton som sen
kan användas till att skicka ut massor av spam! Låt dig alltså inte luras av de
här mailen.

\subsection{Att byta lösenord}

\index{lösenord}
För att inte lösenorden ska bli för lättgissade finns vissa
krav på hur ett lösenord ska se ut. De måste vara minst åtta tecken
långa, och innehålla tecken ur tre av följande fyra grupper:
\begin{itemize}
\item små bokstäver (gemener)
\item stora bokstäver (versaler)
\item siffror
\item specialtecken
\end{itemize}

Lösenordsbyte görs på
%% CHANGE: Ändrade adress till fungerande adress
\url{https://passport.lu.se/}

När man har bytt sitt lösenord kan det ta någon minut innan det fungerar
överallt.

\section{Filer och kataloger}

Hittills har vi bara tittat på grunderna i användargränssnittet. Nu ska vi börja
titta närmare på styrkorna med Linux terminal. Ni får nu bekanta er med en uppsättning
enkla kommandon som man skriver i ett terminalfönster. Varje kommando motsvarar ett program som finns som standard i Unix-liknande operativsystem.
Var för sig är dessa kommandon inte så kraftfulla, men genom att kombinera dem kan man göra
riktigt avancerade saker. (Det här går vi in på mer i detalj i nästa kapitel.)

Det är skillnad mellan stora och små bokstäver, både i filnamn och kommandon, så
tänk på det när ni testar kommandon.

\subsection{Att starta teminalen}
Det finns flera olika sätt att starta terminalen. Ifall du använder en dator med Ubuntu, så kan du öppna Ubuntus sökprogram ''dash''
genom att trycka ned windows-knappen på tangentbordet. När dash syns på skärmen skriver du in ''terminal'' i sökfönstret och startar terminalen.
Snabbknapparna \emph{Ctrl+alt+T} brukar också fungera på de flesta linuxvarianterna.  

När du har lyckats starta en terminal så befinner du dig i din hem-mapp. Med kommandot \emindex{ls}  kan du läsa vad som finns i mappen. 
I följande kapitel kommer vi att gå igenom lite grundkommandon för navigering och filhantering. Gå igenom dem noga ifall du inte redan 
känner dig säker på hur du manövrerar terminalen.

\subsection{Att skapa filer}

Ett sätt att skriva in text i en fil är att använda kommandot
\emindex{cat}. Det är en förkortning av \emph{concatenate} som betyder
att slå ihop. Kommandot kan användas för att slå ihop flera filer, men
i sin enklaste form läser det rader från tangentbordet och skriver dem
till terminalen:
\begin{example}
lo-1\{e01nn\}: \textbf{cat}
\textbf{Hej hopp i lingonskogen}
Hej hopp i lingonskogen
\textbf{Detta är den andra raden av tre}
Detta är den andra raden av tre
\textbf{Nu är det dags att avsluta och gå vidare}
Nu är det dags att avsluta och gå vidare
\textbf{\Ctrl{d}}
lo-1\{e01nn\}: 
\end{example}
För att avsluta inmatningen av text skriver man ett
\emindex{filslutstecken} (\emindex{end-of-file}) först på en rad. I
Linux är det tecknet som heter \Ctrl{d}, som man får fram genom att
hålla nere tangenten märkt\label{control} \textsc{control} eller
\textsc{ctrl} samtidigt som man trycker på tangenten \textsc{d}.
Nu är det ganska meningslöst att skriva text från tangentbordet till
skärmen. Vi vill kunna spara texten i en fil. Det gör man genom att
använda tecknet \ttindex{>} för att tala om för Linux att dirigera om
utmatningen från ett program till en fil:
\begin{example}
lo-1\{e01nn\}: \textbf{cat > nisse}
\textbf{Nils Nilsson
Gatvägen 12
123 45 Lundfanstorp}
\textbf{\Ctrl{d}}
lo-1\{e01nn\}: 
\end{example}
Här ser man att terminalen inte upprepar varje mening (som den gjorde i förra exemplet). Detta är för att
utskriften från cat dirigeras om (med hjälp av '>') till filen Nisse, istället för att skrivas ut i fönstret.  
Svenska bokstäver\index{svenska bokstäver} och bokstäver med accenter
ska man helst undvika i filnamn. Mycket av utvecklingen sker i USA, och det är
inte alltid utvecklarna tänker på att programmen ska fungera med våra
bokstäver.


\subsection{Att titta i filer}

För att skriva ut innehållet i en fil på skärmen använder man också
programmet cat. Genom att skriva ett filnamn efter cat läser
programmet från filen i stället för från tangentbordet:
\begin{example}
lo-1\{e01nn\}: \textbf{cat nisse}
Nils Nilsson
Gatvägen 12
123 45 Lundfanstorp
lo-1\{e01nn\}: 
\end{example}
\subsection{Bläddra i filer}

Nu har vi skrivit ut filer på skärmen med cat. Det har nackdelen att man
inte kan titta på stora filer. Prova till exempel att skriva ut systemets
ordlista med cat:

\begin{example}
lo-1\{e01nn\}: \textbf{cat /usr/share/dict/words}
\end{example}

För att kunna titta på en så stor fil finns programmet \emindex{less}. För att
kunna titta på systemets ordlisa utan att den rullar av skärmen skriver vi:

\begin{example}
lo-1\{e01nn\}: \textbf{less /usr/share/dict/words}
\end{example}

Nu kan vi gå framåt i texten, sida för sida, med \emindex{mellanslag} eller
\emindex{f}, och bakåt med \emindex{b}. Man kan även scrolla i texten med hjlp av piltangenterna. För att avsluta less trycker man \emindex{q}.
less har inbyggd hjälp som man får om man trycker \emindex{h}.
Less är inte standard i alla Unix-liknande system, men det finns alltid ett
program som heter \emph{more} som fungerar ungefär som less men som
inte har så många finesser.

\subsection{Att lista filer}

Om man inte kommer ihåg vad man kallade filen kan man använda
kommandot \emindex{ls} vilket är en förkortning för list. \emph{ls}
skriver ut namnen på alla filer:
\begin{example}
lo-1\{e01nn\}: \textbf{ls}
Desktop/ Documents/ Downloads/ Pictures/\\Videos/   nisse
lo-1\{e01nn\}: 
\end{example}
Filer som slutar med snedstreck, t.ex. \emph{Documents/} är en katalog som systemet har skapat åt
dig. Vi åter kommer till dessa senare.

\subsection{Att flytta, döpa om och ta bort filer}

Nu ska vi titta på tre kommandon som heter \emindex{cp}, \emindex{mv}
och \emindex{rm}. Det är förkortningar för \emindex{copy}, \emindex{move} respektive
\emindex{remove}.

Med kommandot
\begin{example}
lo-1\{e01nn\}: \textbf{cp nisse nisse2}
\end{example}
skapar man en kopia av filen nisse som heter nisse2. Om man i stället
skriver
\begin{example}
lo-1\{e01nn\}: \textbf{mv nisse nisse2}
\end{example}
döper man om nisse till nisse2. Nu finns alltså inte filen nisse kvar.
Genom att skriva
\begin{example}
lo-1\{e01nn\}: \textbf{rm nisse2}
\end{example}
tar man bort filen nisse2.
\subsection{Kataloger}

\index{katalog}\index{filkatalog}
Alla filer lagras i \emph{kataloger} (på engelska heter det
\emindex{directory}). I en katalog kan det finnas
\rmindex{underkataloger} i flera nivåer. Man brukar kalla det hela för
ett \rmindex{katalogträd} eftersom det ser ut som ett träd (med roten
uppåt).

Varje användare har en egen \emindex{hemmakatalog}. Det är där man
befinner sig när man precis har loggat in. De filer som vi skapade i
den förra övningen låg alla i hemmakatalogen.
\begin{figure}[tbp]
  \centerline{\begin{picture}(80,28)(-40,-25)
      \put(0,0){\makebox(0,0){\ST{\~}}}
      \put(0,0){\oval(20,6)}
      \put(7,-3){\line(2,-1){14}}
      \put(0,-3){\line(0,-1){6}}
      \put(-7,-3){\line(-2,-1){12}}
      \put(-25,-12){\makebox(0,0){\ST{Mail}}}
      \put(-25,-12){\oval(20,6)}
      \put(0,-12){\makebox(0,0){\ST{adresser}}}
      \put(0,-12){\oval(20,6)}
      \put(21,-12){\makebox(0,0){\ST{nisse}}}
      \put(6,-15){\line(1,-1){7}}
      \put(0,-15){\line(0,-1){7}}
      \put(-6,-15){\line(-1,-1){7}}
      \put(-13,-24){\makebox(0,0){\ST{fil1}}}
      \put(0,-24){\makebox(0,0){\ST{fil2}}}
      \put(13,-24){\makebox(0,0){\ST{fil3}}}
    \end{picture}}
  \caption{Katalogträd för en användare}
  \label{fig:hemma}
\end{figure}

I figur~\ref{fig:hemma} finns ett exempel på hur en användares
katalogträd kan se ut. Anledningen till att själva hemmakatalogen
skrivs som tilde (\ST{\~}) återkommer vi till
senare.


\subsection{Att skapa och ta bort kataloger}

För att skapa en katalog\index{katalog!skapa} använder man kommandot
\emindex{mkdir} som betyder \emph{make directory}. Med \emindex{rmdir}
(\emph{remove directory}) tar man bort katalogen\index{katalog!ta
bort}.

Om man t~ex vill skapa en katalog för att lagra adresser till sina
vänner skriver man:
\begin{example}
lo-1\{e01nn\}: \textbf{mkdir adresser}
\end{example}
Nu kan vi använda \emph{ls} för att se hur det ser ut:
\begin{example}
lo-1\{e01nn\}: \textbf{ls}
Mail/      adresser/  nisse
\end{example}
Lägg märke till att det finns ett
\rmindex{snedstreck}\hidettindex{/} sist på den nya katalogen. Det
ingår inte i filnamnet, utan det är ls som lägger till det för att
visa att det är en katalog och inte en vanlig fil. På samma sätt får
man en asterisk (\ttindex{*}) sist på ett filnamn om filen är
ett exekverbart program (t~ex rm eller ett program som man har gjort själv i
någon kurs).
På nyare system är det även vanligt att olika typer av filer/kataloger har olika färg på den text som visas av ls.

\subsection{Att byta aktuell katalog}

Den aktuella katalogen\index{aktuell katalog}\index{katalog!aktuell}
heter på engelska \emph{working directory}. För att se var i
filsystemet man befinner sig använder man kommandot \emindex{pwd} som
betyder \emph{print working directory}.
\begin{example}
lo-1\{e01nn\}: \textbf{pwd}
/h/d8/a/e01nn
\end{example}

Om man vill byta aktuell katalog använder man kommandot \emindex{cd}
(\emph{change directory}).
\begin{example}
lo-1\{e01nn\}: \textbf{pwd}
/h/d8/a/e01nn
lo-1\{e01nn\}: \textbf{cd adresser}
lo-1\{e01nn\}: \textbf{pwd}
/h/d8/a/e01nn/adresser
lo-1\{e01nn\}: \textbf{cd ..}
lo-1\{e01nn\}: \textbf{pwd}
/h/d8/a/e01nn
\end{example}
Som ni ser använde vi \ST{..} som argument till cd för att gå upp en
nivå i filträdet. Just i detta fall kunde vi ha skrivit \ST{cd} utan
argument. Då kommer man nämligen alltid till hemmakatalogen.

\subsection{Att använda kataloger}

Nu kan vi använda \emindex{cp} och \emindex{mv} för att kopiera eller
flytta filer till den nya katalogen. Om man skriver
\begin{example}
lo-1\{e01nn\}: \textbf{cp nisse adresser}
\end{example}
så upptäcker cp att destinationen är en katalog och skapar en ny fil
som heter likadant som originalet, dvs nisse, men som ligger i
katalogen adresser. Nu kan vi titta på den nya filen
\begin{example}
lo-1\{e01nn\}: \textbf{cat adresser/nisse}
Nils Nilsson
Gatvägen 12
123 45 Lundfanstorp
\end{example}
Som ni ser använder man snedstreck (\ST{/}) för att skilja katalog och
filnamn åt. Därför kan aldrig ett snedstreck ingå i ett filnamn.

Om man vill kalla den kopierade filen något annat än originalet
skriver man t~ex
\begin{example}
lo-1\{e01nn\}: \textbf{cp nisse adresser/nisse2}
\end{example} 
Det fungerar likadant med \emph{mv}, dvs om destinationen är en
katalog får filen samma namn i den nya katalogen.

Man kan även använda \emph{mv} för att döpa om en katalog:
\begin{example}
lo-1\{e01nn\}: \textbf{mv adresser addresses}
\end{example}
För att se innehållet i en underkatalog använder man \emindex{ls} men
lägger till katalogens namn:
\begin{example}
lo-1\{e01nn\}: \textbf{ls addresses}
nisse    nisse2
\end{example}
\subsection{Absoluta och relativa sökvägar}

Katalogträdet i figur~\ref{fig:hemma} är egentligen bara en del av
systemets stora filträd. I figur~\ref{fig:tree} ser vi det insatt i
sitt sammanhang tillsammans med en del andra kataloger i Linuxsystemet.
\begin{figure}[tbp]
  \centerline{\begin{picture}(80,76)(-40,-25)
      \put(0,48){\makebox(0,0){\ST{/}}}
      \put(0,48){\oval(30,6)}
      \put(0,45){\line(0,-1){6}}
      \put(4,45){\line(2,-1){12}}
      \put(-4,45){\line(-2,-1){12}}
      \put(12,45){\line(3,-1){18}}
      \put(-12,45){\line(-3,-1){18}}
%
      \put(-30,36){\makebox(0,0){\ST{var}}}
      \put(-30,36){\oval(10,6)}
      \put(-15,36){\makebox(0,0){\ST{usr}}}
      \put(-15,36){\oval(10,6)}
      \put(15,36){\makebox(0,0){\ST{bin}}}
      \put(15,36){\oval(10,6)}
      \put(30,36){\makebox(0,0){\ST{tmp}}}
      \put(30,36){\oval(10,6)}
      \put(0,36){\makebox(0,0){\ST{h}}}
      \put(0,36){\oval(10,6)}
      \put(-2,33){\line(-5,-1){30}}
      \put(-1,33){\line(-2,-1){12}}
      \put(0,33){\line(0,-1){6}}
      \put(1,33){\line(2,-1){12}}
      \put(2,33){\line(5,-1){30}}
%
      \put(-33,24){\makebox(0,0){\ST{d1}}}
      \put(-33,24){\oval(10,6)}
      \put(-23,24){\makebox(0,0){\ldots}}
      \put(-13,24){\makebox(0,0){\ST{d7}}}
      \put(-13,24){\oval(10,6)}
      \put(13,24){\makebox(0,0){\ST{da}}}
      \put(13,24){\oval(10,6)}
      \put(23,24){\makebox(0,0){\ldots}}
      \put(33,24){\makebox(0,0){\ST{dd}}}
      \put(33,24){\oval(10,6)}
      \put(0,24){\makebox(0,0){\ST{d8}}}
      \put(0,24){\oval(10,6)}
      \put(0,21){\line(0,-1){6}}
      \put(-1,21){\line(-2,-1){12}}
      \put(1,21){\line(2,-1){12}}
      \put(2,21){\line(4,-1){24}}
%
      \put(-12,12){\makebox(0,0){\ST{a}}}
      \put(-12,12){\oval(10,6)}
      \put(0,12){\makebox(0,0){\ST{b}}}
      \put(0,12){\oval(10,6)}
      \put(12,12){\makebox(0,0){\ST{c}}}
      \put(12,12){\oval(10,6)}
      \put(24,12){\makebox(0,0){\ST{d}}}
      \put(24,12){\oval(10,6)}
      \put(-13,9){\line(-3,-1){18}}
      \put(-12,9){\line(1,-1){6}}
      \put(-11,9){\line(5,-1){30}}
%
      \put(-32,0){\makebox(0,0){\ST{d99xx}}}
      \put(-32,0){\oval(20,6)}
      \put(18,0){\makebox(0,0){\ST{c01yy}}}
      \put(18,0){\oval(20,6)}
      \thicklines
      \put(-7,0){\makebox(0,0){\ST{\textbf{e01nn}}}}
      \put(-7,0){\oval(20,6)}
      \put(-14,-3){\line(-2,-1){12}}
      \put(-7,-3){\line(0,-1){6}}
      \put(0,-3){\line(2,-1){14}}
%
      \put(-32,-12){\makebox(0,0){\ST{\textbf{Mail}}}}
      \put(-32,-12){\oval(20,6)}
      \put(-7,-12){\makebox(0,0){\ST{\textbf{adresser}}}}
      \put(-7,-12){\oval(20,6)}
      \put(14,-12){\makebox(0,0){\ST{\textbf{nisse}}}}
      \put(-13,-15){\line(-1,-1){7}}
      \put(-7,-15){\line(0,-1){7}}
      \put(-1,-15){\line(1,-1){7}}
%
      \put(-20,-24){\makebox(0,0){\ST{\textbf{fil1}}}}
      \put(-7,-24){\makebox(0,0){\ST{\textbf{fil2}}}}
      \put(6,-24){\makebox(0,0){\ST{\textbf{fil3}}}}
    \end{picture}}
  \caption{Del av ett katalogträd i ett Linuxsystem}
  \label{fig:tree}
\end{figure}

Längst upp i trädet finns den så kallade \emindex{rotkatalogen} eller
kort och gott roten. Den betecknas med ett ensamt
\rmindex{snedstreck}. I just detta speciella fall är alltså \ST{/}
namnet på en katalog i stället för att skilja katalog- och filnamn åt.

När vi skrev \ST{adresser/nisse} var det vad man kallar en
\emindex{relativ sökväg}\index{sökväg!relativ}, dvs namnet utgår från
den aktuella katalogen. Motsatsen är en \emindex{absolut
sökväg}\index{sökväg!absolut}. Den utgår från roten och samma fil som
ovan heter då \ST{/h/d8/a/e01nn/adresser/nisse}. Sökväg heter förresten
\emph{path}\index{path|see{sökväg}} på engelska.

Den absoluta sökvägen till hemmakatalogen för e01nn är
\ST{/h/d8/a/e01nn}. Anledningen att vi skrev den som \rmindex{tilde}
(\ST{\~}) är att man kan använda förkortningar i filnamn (se
tabell~\ref{tab:forkort}).
\begin{table}[tbp]
  \hidettindex{.}\hidettindex{..}\index{~@\verb/~/}%
  \begin{tabularx}{\linewidth}{|lX|}\hline
    \ST{\~{}} & min egen hemmakatalog. \\
    \ST{\~{}user} &  hemmakatalog för användaren med användarnamnet 
      \emph{user}.\\
    \ST{.} &  Aktuell katalog.\\
    \ST{..} &  Katalogen ovanför aktuell katalog. \\
    \hline
  \end{tabularx}
  \caption{Förkortningar i filnamn}
  \label{tab:forkort}
\end{table}

Med denna nya kunskap kan vi skriva sökvägen till katalogen adresser
på ett par olika sätt:
\begin{ttquote}
  adresser \\
  \~{}/adresser \\
  \~{}e01nn/adresser \\
  /h/d8/a/e01nn/adresser \\
  ./adresser \\
  ../../a/e01nn/adresser
\end{ttquote}


\subsection{Wildcards}
\label{wildcards}
Det finns två tecken som är speciella i filnamn, nämligen
\emindex{asterisk} (\ST{*})\hidettindex{*} och
\emindex{frågetecken} (\ST{?})\hidettindex{?}.
Dessa har en särskild egenskap, nämligen att de kan motsvara många
olika tecken. Därför kallas de för \emindex{jokrar} eller
\emindex{wildcards}. \ST{?} motsvarar ett godtyckligt tecken, medan
\ST{*} motsvarar noll eller flera godtyckliga tecken. Med hjälp av
dessa tecken kan man alltså räkna upp flera filer på en kommandorad
utan att behöva skriva så mycket.

Om vi har följande filer i en katalog:

\begin{tabular}{@{\hspace{5mm}}lll}
  prog1.p    &  prog2.p    & uppgift.txt\\
  prog1.o    &  prog2.o    & prog\\
  profile.txt&  inpris.dat & utpris.dat
\end{tabular}

så kan vi använda kommandona till vänster i stället för de till höger:

\begin{tabular}{@{\hspace{5mm}}l@{\hspace{2mm}}l}
  rm prog?.p &   rm prog1.p prog2.p \\
  rm pr*.*   &   rm profile.txt prog1.o prog1.p prog2.o prog2.p \\
  rm *.o     &   rm prog1.o prog2.o
\end{tabular}

Observera att Linux är gjort för programmerare, av programmerare. Det
betyder bland annat att man anses veta vad gör när man skriver ett
kommando. Skriver man \emph{rm *} så kommer \textbf{alla} filer att
försvinna utan att man får en enda fråga. \textbf{Var alltså försiktig
  med wildcards}.
\section{Hjälp}

\index{hjälp}I Linuxsystem finns det normalt alltid textbaserade
manualer till alla program.
Programmet som används heter \emindex{man}. Om man t~ex vill
lära sig allt om ls skriver man
\begin{example}
lo-1\{e01nn\}: \textbf{man ls}
\end{example}
så får man fram den kompletta referensmanualen\label{man-less} för ls.
Manualen navigeras lättast med piltangenterna och när man är klar trycker man \emph{q} (quit) för att avsluta.

%% CHANGE: Ändrade adress
Det finns även en del hjälp på www.student.lth.se/datorsupport/.
Här hittar man information om e-post, egna websidor och annat, och på
förstasidan finns det driftinformation.

Om ni inte själva lyckas hitta lösningen på ett problem är det dags
att prata med någon som kanske vet mer. Har problemet att göra med en
uppgift som man ska lösa i en kurs, bör man prata med institutionen
som ger kursen. Om man däremot misstänker att det är något fel på
själva datorsystemet ska man prata med \emph{Datordriftgruppen (DDG)}.

\section{Datordriftgruppen}

\index{datordriftgruppen}Datordriftgruppen LTH eller kort och gott
\emindex{DDG} har till uppgift att sköta undervisningsdatorerna på
LTH och en del annat. Totalt är vi 10 personer. 

\subsection{Felanmälan}
Om man vill rapportera ett fel behöver man inte besöka DDG
personligen. Det går också bra att anmäla fel genom att skicka brev till
\ST{support@lth.se}. 

Tänk på att berätta utförligt om felet, då ökar chansen att det blir
åtgärdat. Ta med eventuella felutskrifter från datorn i din
felrapport.

Vi har inte möjlighet att gå runt och kontrollera att alla datorer
fungerar som de ska, så därför är vi beroende av felrapporter för att
kunna åtgärda trasiga datorer.

Tro inte att någon annan redan har rapporterat felet. Om alla
resonerar så kan datorerna vara trasiga i flera veckor innan vi
upptäcker det. Vi tar hellre emot flera anmälningar av samma fel, än
inga alls.


%=============================================================================
\chapter{Påbyggnad}
%=============================================================================
\label{cha:pabyggnad}

\section{Skrivare}

\label{sec:skrivare}\index{skrivare}\index{utskrift}
I de flesta datorrum finns en \rmindex{laserskrivare}.
För att kunna skriva ut så kopplar man sitt kontokort till ett skrivarkonto. 
Sedan drar universitetet pengar allt eftersom utskrifter görs. Detta systemet kallas för PaperCut. Registrering av kontokort gör man på: \\
http://www.print.lu.se/papercut/ \\
Utskriftssystemet är gemensamt för hela Lunds 
unviversitet och går alltså även att använda utanför LTH.


\subsection{Att skriva ut}

Från programmen kan man oftast skriva ut via menyerna, men för att skriva ut
en fil från kommandoraden använder man kommandot \emindex{lp} som står
för line print. För att skriva ut en filen \emph{nisse} skriver man
\begin{example}
lo-1\{e01nn\}: \textbf{lp nisse}
\end{example}
då kommer utskriften på den närmaste skrivaren. Vill man skicka
utskriften till en viss skrivare anger man dess namn:
\begin{example}
lo-1\{e01nn\}: \textbf{lp -d hacke nisse}
\end{example}
Nu kommer filen \emph{nisse} att skrivas ut på skrivaren i rummet
\emph{hacke} (alla skrivare heter likadant som det rum de står i). När
utskriften har skickats iväg får man ett meddelande uppe
i högra hörnet på skärmen.

För att få lite snyggare formattering av utskrifterna kan man använda
programmet \emindex{a2ps}. Till exempel gör man så här för att skriva ut ett
dokument med två sidor på varje papper:
\begin{example}
lo-1\{e01nn\}: \textbf{a2ps -2 nisse}
\end{example}
För att se vad man kan göra med \emindex{a2ps} i övrigt, skriv
\begin{example}
lo-1\{e01nn\}: \textbf{a2ps --help}
\end{example}

Självklart går det bra att skriva ut från andra program som stödjer utskift
så som Adobe Reader eller en texteditor. För att skriva ut från dessa program
så trycker man på \emindex{skriv ut} i programmen och sedan väljer den skrivare
som heter samma sak som den sal man sitter.

\subsection{Skrivarköer}

\index{skrivarkö}När man använder lp skickas inte utskriften direkt
till skrivaren. I så fall skulle man inte kunna skicka iväg utskrifter
förrän en tidigare utskrift var klar. I stället skickar man utskriften
till en annan dator där ett särskilt program, en så kallad
\emindex{skrivardemon} eller \emindex{spooler}, tar hand om den.
Utskriften placeras i en kö och skickas sedan vidare till
skrivaren när den är klar med de tidigare utskrifterna.

Varje skrivare har sin egen kö, så det finns lika många köer som det
finns skrivare i systemet.

Med kommandot \emindex{lpstat} kan man se efter vad
som finns i skrivarkön.

\subsection{Papper}

\index{papper} DDG fyller på skrivarna med jämna mellanrum, men ibland
kan det blir tomt i pappersmagasinen. I så fall är det bara att säga till.

\section{Filnamn}
\label{filnamn}
I Linux kan ett \rmindex{filnamn} se ut nästan hur som helst. Det finns
ingen begränsning på hur långa namnen får vara, och formatet kan se ut
nästan hur som helst. I vissa andra operativsystem måste filnamn se ut
som \emph{namn.typ}, men i Linux behöver man inte skriva så. Det är
dock vanligt att man använder den konstruktionen för att tala om vad
det är för slags fil. Några exempel hittar ni i tabell~\ref{tab:filext}.
\begin{table}[tbp]
  \begin{tabularx}{\linewidth}{|lX|}\hline
    \ST{fil.java}  & Javaprogram \\
    \ST{fil.c}  & C-program \\
    \ST{fil.C}  & C++-program \\
    \ST{fil.cc}  & C++-program \\
    \ST{fil.txt}  & textfil \\
    \ST{fil.ps}  & grafikformatet postscript \\
    \ST{fil.pdf}  & grafikformatet PDF \\
    \ST{fil.tex}  & källkod till typsättningsprogrammet \LaTeX{} \\
    \hline
  \end{tabularx}
  \caption{Vanliga filnamn}
  \label{tab:filext}
\end{table}
För att undvika problem bör man begränsa sig till att använda följande
tecken i filnamn: \emph{a--z, A--Z, 0--9, ., --, \_}. Det är i och för
sig tillåtet att använda t~ex svenska bokstäver i filnamn, men
eftersom många program är skrivna av amerikaner fungerar det inte
alltid.

\subsection{Punktfiler}

Om en fil har en punkt först i namnet brukar den kallas för en
\emindex{punktfil}. Sådana filer syns inte när man skriver
ls. Dessa filer innehåller inställningar till program och annat som
mest är i vägen när man listar sina filer. Det går dock att se
punktfilerna om man skriver \emph{ls -a}.
\section{Kommandon}

Linux skiljer inte på \emindex{kommandon} och
\emindex{program}\index{programstart}. Alla de kommandon som vi har
använt hittills (cat, ls, mv, cp, rm, ...) är fristående program, och
inte på något sätt inbyggda i systemet. Att köra programmet \emph{cat}
fungerar alltså likadant som att köra ett program som man har skrivit
själv. 

\subsection{Format}

De flesta kommandon består av kommandots eller programmets namn, följt
av eventuella \emindex{optioner} och \emindex{argument}. Om vi till
exempel skriver
\begin{example}
lo-1\{e01nn\}: \textbf{ls -l adresser}
\end{example}
så är:
\begin{description}
\item[ls] namnet på programmet som ska köras.
\item[-l] en option som i detta fall talar om att ls ska använda ett
  längre format när den visar innehållet.
\item[adresser] ett argument, i det här fallet namnet på den katalog
  som vi vill se innehållet i.
\end{description}

Vissa program kan använda flera optioner samtidigt. Då kan man oftast
skriva ihop dem. Ordningen mellan optionerna är inte heller viktig för
det mesta. Följade rader är helt ekvivalenta:
\begin{ttquote}
  ls -l -F -C -i adresser \\
  ls -lF -C -i adresser \\
  ls -lFCi adresser \\
  ls -ilCF adresser 
\end{ttquote}
Vissa optioner behöver argument. Det har vi redan stött på med lp där
man kunde skriva
\begin{example}
lo-1\{e01nn\}: \textbf{lp -d hacke nisse}
\end{example}
I detta fallet är \ST{hacke} ett argument till optionen \ST{-d}. I de
flesta fall kan man skriva ihop optioner och deras argument:
\begin{example}
lo-1\{e01nn\}: \textbf{lp -dhacke nisse}
\end{example}
vilket gör att man inte så lätt blandar ihop dem med själva
programmets argument (i detta fallet filnamnet nisse).

Tänk på att det är skillnad mellan små och stora bokstäver även i
optioner. \ST{lp~-dhacke} är inte samma sak som \ST{lp -Dhacke}.

\subsection{Starta program}

För att starta ett program skriver man helt enkelt programmets
namn. Det har vi gjort många gånger nu, med program som \emph{rm}
eller \emph{cat}.

När vi skapade ett nytt terminalfönster eller tog fram en text editor på
skärmen körde vi också program, men i stället för att skriva
programmets namn bad vi fönsterhanteraren att köra programmet åt oss
genom att välja en rad i en meny. Man kan själv starta nya fönster
genom att skriva programmets namn i ett terminalfönster.

Om man startar program så finns det en risk att terminalfönstret låser sig
tills programet har avslutats. Detta löser man genom att starta program i 
\emindex{bakgunden} och detta gör man genom att skriva tecknet \& efter 
kommandot.
\subsection{Var finns programmet?}

Programmet cat består av en fil som heter cat. Men det finns ingen
sådan fil i din hemmakatalog. Hur bär sig då Linux åt för att hitta
alla program?

I en variabel som heter \emph{PATH} lagras sökvägen\index{sökväg} till
ett antal kataloger där systemet letar efter program.


Som ni ser så finns det rätt många kataloger med program. Bland dem
finns några som ligger i hemmakatalogen. Där kan man lägga program som
man vill kunna köra utan att skriva fullständig sökväg.

Observera att den katalog man befinner sig i \emph{inte} ingår i
sökvägen. Vill man köra ett program  som ligger i den
aktuella katalogen måste man skriva \emph{./programnamn}.

\section{Omdirigering av I/O}

Först några ord om rubriken. \rmindex{I/O} betyder Input/Output, dvs in-
och utmatning till eller från ett program.

Alla program har tillgång till tre kanaler för I/O, en inkanal och två
utkanaler. Inkanalen kallas \emindex{stdin} (standard input) och är
normalt kopplad till tangentbordet. Utkanalerna heter \emindex{stdout}
(standard output) och \emindex{stderr} (standard error) och är normalt
kopplade till terminalfönstret. På stderr kommer felutskrifter och på
stdout kommer normal utmatning från ett program.

När man skriver ett kommando kan man med skrivsätten i
tabell~\ref{tab:io-redir} ange att dessa tre kanaler ska kopplas någon
annanstans än till tangentbord eller skärm.
\begin{table}[tbp]
  \hidettindex{<}\hidettindex{>}\hidettindex{>>}\hidettindex{2>}%
  \index{>"&@\verb+>"&+}\index{"\"|@\verb+"|+}%
  \begin{tabularx}{\linewidth}{|lX|}\hline
    \emph{> fil}  & Koppla stdout till \emph{fil}. Filen skapas om den
    inte redan finns. Om filen redan finns skrivs den över.\\ 
    \emph{>{}> fil} & Som ovan, men en existerande fil skrivs inte
    över. I stället läggs den nya utmatningen till i slutet av filen. \\
    \emph{2> fil} & Koppla stderr till \emph{fil}. \\
    \emph{>\& fil} & Koppla både stdout och stderr till \emph{fil}. \\
    \emph{< fil} & Koppla stdin till \emph{fil}. Gör att ett program
    läser sin inmatning från \emph{fil}. \\
    \emph{$\mid$ prog} & koppla stdout till stdin för programmet
    \emph{prog}. \\
    \emph{$\mid$\& prog} & koppa stdout och stderr till stdin för programmet
    \emph{prog}. \\
    \hline
  \end{tabularx}
  \caption{Sätt att dirigera om I/O}
  \label{tab:io-redir}
\end{table}
Den första formen av omdirigering har vi redan stött på när vi skrev till
filer med cat, men resten kräver nog en del förklaring. För att göra
detta ska vi introducera ett nytt program som heter \emindex{wc} (word
count). Det räknar tecken, ord och rader i en fil:
\begin{example}
lo-1\{e01nn\}: \textbf{wc nisse}
       3       7      45 nisse
\end{example}
Utmatningen berättar att filen \emph{nisse} innehåller 3 rader, 7 ord
respektive 45 tecken. Nu ska vi dirigera om utmatningen från wc till
en fil:
\begin{example}
lo-1\{e01nn\}: \textbf{wc nisse > utfil}
lo-1\{e01nn\}: \textbf{cat utfil}
       3       7      45 nisse
\end{example}
I detta fallet hamnade utmatningen från programmet wc i filen
utfil. Men det är bara stdout vi har dirigerat om, inte stderr, vilket
vi kan se om vi försöker räkna ord i en fil som inte finns:
\begin{example}
lo-1\{e01nn\}: \textbf{wc finns-inte > utfil}
wc: cannot open finns-inte
lo-1\{e01nn\}: \textbf{cat utfil}
lo-1\{e01nn\}: 
\end{example}
Som ni ser fick vi en tom utfil eftersom wc inte skrev något på
stdout.

wc läser normalt sin inmatning från tangentbordet (tills vi avslutar
med \Ctrl{d}). Hittills har vi dock gett wc ett filnamn som
argument för att den ska läsa från filen i stället. Alla program har
dock inte denna möjlighet, och då kan man dirigera om inmatningen så
att den kommer från en fil:
\begin{example}
lo-1\{e01nn\}: \textbf{wc < nisse}
       3       7      45
\end{example}
Nu skrev inte inte wc ut något filnamn eftersom programmet inte vet
vad filen heter.

\subsection{Pipe}

Den båda sista formerna av omdirigering i tabell~\ref{tab:io-redir} är lite
speciella. Tecknet $\mid$ brukar kallas för \emindex{pipe} eftersom det
skapar en slags rörledning för data.

När programmet \emindex{ls} skriver till skärmen får man filnamnen
uppställda i kolumner. Om man dirigerar om utmatningen skriver
\emindex{ls} i stället bara ett filnamn per rad. Det kan
man använda för att räkna filer.

Om vi till exempel vill räkna antalet filer i en katalog kan vi
använda en temporär fil, och skriva
\begin{example}
lo-1\{e01nn\}: \textbf{ls > tmp}
lo-1\{e01nn\}: \textbf{wc < tmp}
       7       7      48
lo-1\{e01nn\}: \textbf{rm tmp}
\end{example}
Det blir rätt mycket att skriva för en sådan enkel uppgift.
Dessutom måste vi skapa en fil och vänta med att räkna orden tills ls
är klar med att lista dem. Sen blir det en fil för mycket i
sammanställningen eftersom den temporära filen också räknas.

I stället kan man med en pipe koppla ihop två kommandon. Tecknet
$\mid$ anger att stdout från det första programmet ska kopplas ihop
med stdin för det andra:
\begin{example}
lo-1\{e01nn\}: \textbf{ls | wc}
       6       6      44
\end{example}
Detta är mycket enklare att skriva och har en finess som inte är helt
uppenbar, nämligen att båda programmen körs samtidigt. Om man kopplar
samman två program som tar lång tid kan det gå betydligt fortare att
använda en pipe än en temporär fil.

Man kan givetvis koppla ihop fler än två program och dessutom
kombinera det med att läsa från filer:
\begin{example}
lo-1\{e01nn\}: \textbf{cat < nisse | wc | cat > utfil}
lo-1\{e01nn\}: \textbf{cat utfil}
       3       7      45
\end{example} 
Program som cat och wc läser sin inmatning från stdin och skriver sin
utmatning till stdout. Program av denna typen lämpar sig väl för
sammankoppling med en pipe, och brukar kallas för \emindex{filter}.
\section{E-post}
\index{e-post}\index{email}
På Lunds universitet används gmail till e-post. (inloggning sker på http://mail.google.com)
Din E-postadress är \emph{förnamn.efternamn.nnn@student.lu.se} där \emph{nnn}
är tre slumpmässiga siffror.
Det går också bra att ange stil-id vid inloggning. (användarnamnet du fick för att logga in på datorerna)
ange då: \\ \textbf{stil-id@student.lu.se} \\
i ''email or phone''-fältet, så hänvisas du vidare till Lunds universitets
inloggningsportal. Därefter anger du enbart stil-id när du fyller i användarnamn.

\subsection{Läs dina brev}

E-postadressen är studentens officiella adress. Lärare och administratörer
på LTH använder den adressen för att komma i kontakt med studenterna.
Därför är du \textbf{skyldig att bevaka din e-post hos oss}, antingen
genom att läsa den i webmailen eller genom att vidarebefordra den till en
annan adress. Fråga din övningsledare om du vill ha hjälp att forwarda din e-post till just din adress.

\section{Forum}

LTH har ett eget forum för studenterna. Det återfinns på
http://forum.student.lth.se/. Här kan man diskutera allt som har med studierna
att göra. DDG läser på forumet och svarar på frågor om datorer.

\section{Information om filer}

\index{filer!information}Genom att ge programmet \emindex{ls} optionen
\emph{-l} listar den filerna i ett annat format:
\begin{example}
\scriptsize
lo-1\{e01nn\}: \textbf{ls -l nisse}
-rw-------   1 e01nn  students  45  Jul 28 16:58 nisse
\end{example}
Den första kolumnen är information om filskyddet. Det återkommer vi
till på sidan~\pageref{sec:filskydd}. Sedan följer en etta som vi
bortser från (skriv \emph{man ls} om du är intresserad). Därefter
följer filens ägare, gruppen som filen tillhör, filens storlek, datum
då den senast ändrades och filens namn.

\section{Säkerhetskopiering}

Med jämna mellanrum görs en \rmindex{säkerhetskopiering}
(\emindex{backup} på engelska) av hela datorsystemet. Då skrivs alla
ändrade filer till en bandspelare eller till en server med mycket disk.
Kopieringen görs i första hand för
att kunna återställa systemet om ett skivminne skulle gå sönder. I mån
av tid hjälper DDG till med att läsa tillbaka filer som en användare
''råkat radera'' eller som ''bara försvann''. Det är dock ingenting
som ni kan räkna med, så om ni har viktiga filer som ni absolut inte
vill bli av med kan det vara klokt att göra en kopia på ett USB-minne.

För att be att få tillbaka en borttappad fil, skicka ett brev till 
\ST{support@lth.se} och tala om vad filen heter, när den försvann och
när den senast ändrades.
\index{backup}

\section{Komprimering av filer}

Varje användare har ett begränsat utrymme tillgängligt för att lagra
filer. Hur mycket det är, och hur mycket man har använt kan man ta
reda på med ett program som heter \emindex{quota}:
\begin{example}
\scriptsize
lo-1\{e01nn\}: \textbf{quota}
Disk quotas for e01nn (uid 25002): 
Filesystem usage quota limit timeleft files quota limit
/a/d8       2096 25000 50000            137 10000 20000            
\end{example}
Först finns siffrorna för det utrymme som filerna tar upp (räknat i
kilobyte, dvs 1024 tecken), därefter kommer en grupp siffror som
gäller antalet filer. Skillnaden mellan \emph{quota} och \emph{limit}
ligger i att man tillfälligt får gå upp till den övre gränsen. För
vårt system ligger tidsgränsen på 12 timmar.

Om man har gått över den tillåtna gränsen så kan man inte längre logga
in på normalt sätt. Genom att markera \emph{Starta
  endast enkelt fönster} nere i vänstra hörnet på inloggningsskärmen
kan man logga in tillfälligt för att städa.

Tänk på att program som du har kompilerat tar stor plats. Däremot tar
inte källkoden (programtexten) upp mycket utrymme. Ta bort alla
exekverbara program när programmeringsuppgifterna är färdiga så sparar
du mycket utrymme. Detta gäller för de flesta programmeringsspråken,
dock inte Java.

Om man har mycket ont om plats och inte kan städa bort något kan man
komprimera\index{filer!komprimera}\index{komprimera filer} en del av
sina filer. Det gör man med ett program som heter \emindex{gzip}. Det
brukar minst kunna halvera storleken på filerna. För att komprimera
filen \emph{nisse} skriver man
\begin{example}
lo-1\{e01nn\}: \textbf{gzip nisse}
\end{example}
Filen ersätts av den komprimerade filen som i detta fallet heter
\emph{nisse.gz}. Den kan man inte använda som vanligt utan att först
dekomprimera den. För det ändamålet använder man programmet
\emindex{gunzip}:
\begin{example}
lo-1\{e01nn\}: \textbf{gunzip nisse}
\end{example}
Om man vill komprimera eller dekomprimera alla filer i en underkatalog
(och dess underkataloger) lägger man till optionen \emph{-r} (recurse)
till gzip respektive gunzip:
\begin{example}
lo-1\{e01nn\}: \textbf{gzip -r adresser}
\end{example}
En komprimerad textfil kan man titta på utan att packa upp den. Det
gör man med \emindex{zless} eller \emindex{zmore}. Båda programmen gör
exakt samma sak, nämligen packar upp filen tillfälligt och visar den i
less.
\begin{example}
lo-1\{e01nn\}: \textbf{zmore nisse}
\end{example}

Observera att den diskkvot som en användare har ska räcka till alla
filer som produceras under studietiden. Bara i undantagsfall kan man
få mer utrymme att tillgå och i så fall är det bara tillfälligt medan
man läser kurser som kräver mycket plats.

Ett bra sätt att ta reda på hur mycket plats ens filer tar är att
använda programmet \emindex{du} (disk usage):
\begin{example}
lo-1\{e01nn\}: \textbf{du}
4       ./Mail/inbox
2       ./Mail/drafts
14      ./Mail
5       ./adresser
2       ./temp
1       ./Calendar
1       ./News
22      ./.mozilla
645     .
\end{example}
För varje katalog talar \emph{du} om hur många kilobyte filerna tar,
och sist av allt kommer summan för alla kataloger.

\section{Filskydd}

\label{sec:filskydd}\index{filskydd}\index{filer!filskydd}
Nu ska vi återkomma till utmatningen från \emph{ls~-l}:
\begin{example}
\scriptsize
lo-1\{e01nn\}: \textbf{ls -l nisse}
-rw-------   1 e01nn  students  45  Jul 28 16:58 nisse
\end{example}
Den första kolumnen består av 10 tecken. Det första anger filens typ
(de vanligaste är -- för vanliga filer och \emph{d} för kataloger). De
övriga nio tecknen anger hur filskyddet ser ut.

I Linux finns det nämligen möjlighet att skydda filer så att
obehöriga inte kan komma åt dem. För varje fil kan man specificera
huruvida filen får läsas, skrivas eller exekveras av användare i olika
kategorier. Dessa kategorier är:
\begin{description}
\samepage
\item[u (user)] Filens ägare
\item[g (group)] Medlemmar i samma grupp som ägaren. Alla studenter
  är med i samma grupp (\emph{students}).
\item[o (others)] Övriga användare
\end{description}
De olika rättigheterna anges med följande bokstäver
\begin{description}
\samepage
\item[r (read)] Tillstånd att läsa
\item[w (write)] Tillstånd att skriva 
\item[x (execute)] Tillstånd att exekvera (köra) ett program.
\end{description}

Filskyddet anges i ordningen: användare, grupp och övriga. T~ex
betyder \ST{-rw-r-{}-{}-{}-{}-} att ägaren får läsa och skriva filen, att
gruppen bara får läsa och att övriga inte har tillgång till filen.
(Strecket betyder att motsvarande position är avstängd och finns bara
för att det ska vara lättare att läsa filskyddet.)

För att ändra filskyddet använder man kommandot \emindex{chmod} som
tar två argument:
\begin{ttquote}
  chmod kategorier=rättigheter filnamn
\end{ttquote}
Kategorier och rättigheter anges med bokstäverna som förklarats
ovan. Rättigheter kan vara tomt om man vill stänga av alla rättigheter
för en kategori. Om man t~ex vill hindra andra än ägaren att komma åt
en fil skriver man:
\begin{example}
lo-1\{e01nn\}: \textbf{chmod go= nisse}
\end{example}

Nya filer skapas alltid så att de kan exekveras av alla, men
bara läsas och skrivas av ägaren.

\subsection{Kataloger}

För kataloger\index{katalog!filskydd} betyder filskyddet inte riktigt
samma sak som för vanliga filer:
\begin{description}
\item[r] Tillstånd att läsa i katalogen, dvs se vilka filer som finns
  i katalogen (t~ex med ls).
\item[w] Tillstånd att skriva i katalogen, dvs skapa filer och ta bort
  filer. Det är katalogens rättigheter och inte filens som avgör om en
  användare kan ta bort en fil.
\item[x] Tillstånd att ha med katalogen i en sökväg. Om en katalog
  bara har \emph{x} för en viss kategori kan användarna i den
  kategorin fortfarande komma åt filer i katalogen om de vet vad
  filerna heter.
\end{description}


%=============================================================================
\chapter{Överkurs}
%=============================================================================
\label{cha:overkurs}

\section{Java i terminalen}
\index{Java i terminalen}
Om man vill köra ett program som är i skrivet i java så behöver man först kompilera det.
För att kompilera filen \texttt{Main.java} så skriver man:
\begin{example}
lo-1\{e01nn\}: \textbf{javac Main.java}
\end{example}
och man kommer då få en fil som heter \texttt{Main.class}.
För att köra programmet skriver man:
\begin{example}
lo-1\{e01nn\}: \textbf{java Main}
\end{example}

Är det så att man programmet består av mer än en källfil
så behöver man kompilera dessa också. Om man till exempel har ett program
som består av filerna \texttt{Main.java} och \texttt{Network.java} så
kompilerar man dessa tillsammans genom att skriva:

\begin{example}
lo-1\{e01nn\}: \textbf{javac Main.java Network.java}
\end{example}

Har man många java-filer som man vill inkludera så kan det vara lättast att
använda sig av ett wildcard. Det skulle kunna se ut på följande sätt:
\begin{example}
lo-1\{e01nn\}: \textbf{javac *.java}
\end{example}
Då kommer alla java-filer i mappen att kompileras.


\section{Specialtecken}

\index{specialtecken}På sidan \pageref{control} berättade vi lite om
kontrolltecken (som man får fram om man håller nere tangenten
\textsc{control} och trycker på en annan tangent). När man skriver
kommandon i ett terminalfönster har dessa tecken (och en del andra) en
speciell betydelse.

Om man t~ex skriver fel så kan man använda nedanstående kommandon för
att rätta sina fel innan man trycker \emph{Return}:
\begin{description}
\item[Del] Det tecken man senast skrivit raderas. Med upprepade
  \emph{Del} kan man radera flera tecken.
\item[Control-u] Hela raden raderas.
\item[Control-w] Ordet \emph{före} markören raderas.
\item[Control-k] Allt \emph{efter} markören raderas.
\end{description}
Man kan använda piltangenterna $\rightarrow$ och $\leftarrow$ för att
flytta markören på raden. Om man vill köra gamla kommandon igen kan
man använda $\uparrow$ för att få tillbaka dem på kommandoraden. För
varje gång man trycker $\uparrow$ går man ett steg tillbaka i listan
med gamla kommandon. Om man går för långt kan man trycka $\downarrow$
för att gå åt andra hållet.

Man kan även söka efter gamla kommandon om man trycker \emph{Control-r} och
sedan de tecken man söker efter. När man har hittar rätt kommando trycker man
\emph{Return} för att köra det.

Följande kontrolltecken är också bra att känna till:
\begin{description}
\item[Control-c] Avbryter exekveringen av pågående program. Det kan
  användas när man ångrar sig efter att ha tryckt \emph{Return}, eller
  om ett program har hängt sig.
\item[Control-z] Stoppar exekveringen av ett kommando
  tillfälligt. Genom att skriva \emindex{fg} (foreground) startar man
  det igen. Man kan även skriva \emindex{bg} (background). Då
  fortsätter programmet att köra i bakgrunden, precis som om man hade
  startat det med \emph{\&} sist på raden.
\item[Control-d] Filslutstecken. Anger att data är slut vid inmatning
  från tangentbordet.
\item[Control-l] Tömmer hela terminalfönstret på text.
\end{description}

Vissa tangenter har extra symboler som t~ex \{, \} eller [. Dessa får
man fram genom att hålla nere tangenten märkt \emph{Alt Gr}
samtidigt som kan trycker på respektive tangent.

\subsection{Filnamnskomplettering}

\index{filnamn!komplettering}Att använda långa filnamn är bra eftersom
man kan beskriva innehållet i en fil på ett entydigt sätt. Det är
däremot jobbigt att skriva in det långa namnet när man ska använda
filen.

Då kan man använda sig av filnamnskomplettering. Om man skriver början
av filnamnet och sedan trycker på tabulatortangenten (märkt \emph{Tab}
eller $\rightarrow\!\mid\,$) fylls resten av filnamnet i automatiskt. Om
det man skrev inte var unikt visas en lista över alternativ. Då kan man
antingen skriva några tecken till och trycka på \emph{Tab} igen, eller
omdelelbart trycka på \emph{Tab} för att gå igenom alla möjliga alternativ.
tecken till och sedan trycka \emph{Tab} igen.

\section{Alias}

Genom att definiera \emindex{alias} kan man skapa egna förkortningar
av långa kommadon. Alias definieras genom att skriva
\begin{ttquote}
  alias namn='sträng'
\end{ttquote}
Det får inte finnas några blanktecken före eller efter
likhetstecknet. Om man skriver \emph{alias} utan argument får man en
lista över alla alias som finns definierade.

Om man t~ex vill kunna skriva \emph{dir} i stället för \emph{ls~-l}
kan man definiera ett alias:
\begin{example}
  lo-1\{e01nn\}: \textbf{alias dir='ls -l'}
\end{example}

Ett alias finns bara kvar tills man loggar ut,
men genom att lägga in dem i initieringsfilen \emindex{.bashrc} kan man
får dem att definieras varje gång man loggar in (se nedan för
mer info).

\section{Initieringsfiler}

Många program läser en \emindex{initieringsfil} när de startas. Det
finns initieringsfiler av två slag: dels filer som innehåller
kommandon som exekveras när programmet startas, dels filer som
innehåller parametrar som läses av programmet. De vanligaste
initieringsfilerna är:
\begin{description}
\item[.bash\_login] \index{bash\_logine@.bash\_login}Exekveras vid inloggning.
\item[.bash\_logout] \index{bash\_logout@.bash\_logout}Exekveras vid utloggning.
\item[.bashrc] \index{bashrc@.bashrc}Exekveras när man startar ett nytt
  terminalfönster.
\end{description}

Om man inte har dessa filer själv används de som finns i katalogen
\ST{/usr/local/adm} (de har samma namn fast utan inledande punkt). I
vissa fall ersätter de egna filerna den gemensamma och i andra fall är
de tillägg till de gemensamma. Ni som är intresserade kan titta på de
gemensamma filerna och läsa manualerna till \emph{bash}, \emph{X} och
\emph{twm}.

Ett varningens ord är på sin plats: Om man har egna initieringsfiler kan man
få problem. I det läget får man tyvärr räkna med att klara sig själv eftersom
DDG har inte resurser till att sätta sig in i alla personliga konfigurationer.
Kopiera alltså inte kompisens fräcka initieringsfiler om du inte känner att du
har kunskaper nog att ta hand om de problem som kan dyka upp.

\section{Kommandotolken}

Det program som läser och utför kommandon i ett terminalfönster kallas
för \emindex{kommandotolk} eller \emindex{shell}. Den variant som vi
använder heter \emindex{bash}.

Alla initieringsfiler som började med \emph{.bash} i tabellen ovan läses
av \emph{bash} vid inloggning och/eller vid start av en ny
kommandotolk. I de gemensamma initieringsfilerna finns information om
vilken typ av saker som är lämpliga att lägga i vilka filer.

Det vanligaste är att man vill ha en egen \emindex{.bashrc} för att
spara några egna alias och inställningar till bash. Här är ett exempel
på inställningar som kan vara bra att ha i sin \emindex{.bashrc} om man
är nybörjare:
\begin{ttquote}
  alias rm='rm -i' \\
  alias cp='cp -i' \\
  alias mv='mv -i' \\
  setopt noclobber\index{setopt}\index{noclobber} \\
  setopt ignoreeof\index{ignoreeof}
\end{ttquote}
De tre första raderna gör att \emindex{rm}, \emindex{cp} och
\emindex{mv} frågar innan de tar bort eller skriver över en fil.
\emph{setopt noclobber} gör att man inte kan skriva över en fil genom
att dirigera om I/O. \emph{setopt ignoreeof} gör att man inte kan
avsluta ett terminalfönster med \emph{Control-d} utan måste skriva
\emindex{exit} eller \emindex{logout}.

%\section{Institutionernas \\ initieringskommandon}
%
%I många kurser utnyttjas program som behöver särskilda
%inställningar. För att man ska komma åt dessa på ett enkelt sätt utan
%att riskera att de stör övriga inställningar, finns det ett
%initieringskommando för varje institution. De har ett namn som inleds
%med \emph{init} och avslutat med en förkortning för institutionen.
%T~ex skriver man \ST{initcs} för att köra de initieringar som
%behöver för kurser i Datavetenskap.
%
%Lägg absolut inte in dessa kommandon i \emindex{.zshrc} eller någon
%annan initieringsfil. Det kan ställa till stora problem eftersom de
%olika institutionernas initieringskommandon kan komma i konflikt med
%varanandra.

\section{Inloggning på andra \\ datorer}

Ibland kan man behöva logga in på en annan dator än den man sitter
vid. Det kan vara fallet om man sitter hemma, eller i ett lab på någon
institution. För detta ändamål
använder man något av programmet \emindex{ssh} (secure shell client) genom att
skriva:
\begin{ttquote}
  ssh datornamn \\
\end{ttquote}

Tänk på att man absolut inte får köra tunga program på andra datorer
på detta sättet. Då kan man störa den som sitter vid den andra datorn.
I värsta fall kan det medföra att en laboration inte går att
genomföra. Därför har vi ett antal maskiner
som endast är avsedda för
inloggningar utifrån. 

Om man ansluter sig från en dator som befinner sig utanför vårt nätverk så
måste man dessutom använda login-maskinerna. Datorerna i salarna är nämligen
inte tillgänliga utifrån. Man når loginmaskinerna under ett gemensamt namn,
\texttt{login.student.lth.se}.
Hemifrån måste man alltså skriva:
\begin{ttquote}
  ssh username@login.student.lth.se 
\end{ttquote}
För mer information om inloggning hemifrån, se bästa kapitel. 

Man kan ge ett program lite lägre prioritet genom att skriva
\emindex{nice} först på kommandoraden. Det är att rekommendera om man
kör program på någon annan dator än den som man sitter
vid. För att t~ex räkna ord i en fil utan att störa andra på datorn så
mycket skriver man
\begin{example}
lo-1\{e01nn\}: \textbf{nice wc nisse}
       3       7      45 nisse
\end{example}

Om datorn känns ovanligt långsam och man misstänker att någon annan
också kör på den kan man använda kommandot \emindex{who} för att
kontrollera vem som är inloggad på datorn. Tänk dock på att den
databas som who använder kan bli korrupt ibland, så risken finns att
man kan skälla på fel person.

\section{Filöverföring}

\label{ftp}\index{filöverföring}Alla datorer har tillgång till samma
filer, men om man vill flytta filer till eller från en dator i ett
annat datorsystem kan man använda programmet \emindex{sftp} (secure file
transfer protocol). Man startar en överföring med kommandot
\begin{ttquote}
  sftp datornamn
\end{ttquote}
När förbindelsen är upprättad får man ange sitt användarnamn och
lösenord på den andra datorn. Därefter kan man använda kommandon som cd, ls,
get och put. Skriv \emph{man sftp} för mer info.
sftp kör man mot någon av loginmaskinerna.

\section{Processkontroll}

\index{processer}En dator med Linux kör flera program samtidigt genom
att låta varje program exekvera några millisekunder, och sedan byta
till nästa program, eller process som det också kallas.

För att se vilka processer man har startat kan man använda programmet
\emindex{ps} (process status):
\begin{example}
lo-1\{e01nn\}: \textbf{ps}
   PID TTY      TIME CMD
 10083 pts/2    0:00 ps
 10073 pts/2    0:01 xcalendar
 10061 pts/2    0:01 bash
\end{example}
Den första kolumnen visar processens nummer, och den sista det
kommando man skrev.

Om man vill ta bort en process som har hängt sig använder man
kommandot \emindex{kill}. Om vi t~ex vill ta bort den process som kör
kalendern skriver vi:
\begin{example}
lo-1\{e01nn\}: \textbf{kill 10073}
\end{example}
Om processen fortfarande inte vill dö kan man ta i lite mer:
\begin{example}
lo-1\{e01nn\}: \textbf{kill -9 10073}
\end{example}
För mer information, titta på manualerna till ps och kill.

\section{Ordbehandling}

Man brukar skilja på editering och \emindex{ordbehandling}. Editering
handlar om att redigera rena textfiler, medan ordbehandling betyder
att man skriver formaterad text, ofta med olika teckensnitt.

\subsection{\LaTeX}

\LaTeX{} fungerar så
att man skriver sin text med formateringskommandon i en vanlig
editor. Sedan formaterar man den med kommandot \emindex{latex}. Nu
kan man studera utseendet på skärmen med \emph{xdvi} eller skriva ut
den på en laserskrivare med \emph{dvips}. \LaTeX{} är mer eller mindre
standard som ordbehandlare på tekniska universitet i hela världen.  (Den
här handledningen är skriven med \LaTeX.)

Vi har en lokal handledning som heter \emph{Att skriva rapporter med
\LaTeX}. Den används som kurslitteratur i ettan för C och D, men finns även 
att köpa hos DDG för 10 kronor.

\LaTeX{} är fri programvara, och finns för nästan alla datortyper, så
den kan man köra även på sin hemdator.

\subsection{LibreOffice}

För er som trivs bättre med ett system som liknar \emph{Microsoft Word}
så finns \emindex{LibreOffice} tillgängligt i menyerna. LibreOffice är helt
gratis och klarar i allmänhet av att läsa Microsofts dokument. Det kan med
andra ord även vara ett alternativ för hemdatorn. LibreOffice finns även för
Windows och MacOS.


%=============================================================================
\chapter{Hemifrån}
%=============================================================================
\label{cha:hemifran}

Detta kapitel innehåller lite tips om hur man kan utnyttja datorsystemet
hemifrån, samt lite allmänt matnyttig information. Texten bygger till 
delar på ett tidigare häfte gjort av Erik Månsson och Emma Nilsson-Nyman.

\section{E-post hemifrån}

Eftersom Lunds universitets studentmail är webbaserad, så är det enkelt att
sköta sin e-post från vilken dator som helst.


\section{Komma åt filer}

För att titta på, ladda hem eller ladda upp filer så använder man \emph{scp}
(secure copy) eller \emph{sftp} (secure file transfer protocol). Kör man något
Unix-liknande operativsystem (Linux, BSD, MacOS / OSX) hemma har man redan dessa
program. 

Om man vill ladda upp en fil som heter \emindex{Main.java} så kan man göra följande:

\ST{scp ./Main.java username@login.student.lth.se:}


För de som kör windows rekommenderar vi ett program som heter \emph{winscp}.
Det är gratis och kan laddas ned från \ST{http://winscp.net/}. Man kan även
använda ett annat program som heter Filezilla
(\ST{http://filezilla-project.org/}).

Det går även att komma åt filer med FTPS . Då kör man i stället mot en server
som heter ftp.student.lth.se. (I vissa programvaror kan det här även kallas
FTPES, FTP Secure eller FTP-SSL.)


\section{Köra program hemifrån}

Det är givetvis bäst att installera de program man behöver på sin hemdator,
men ibland går det inte. Det kan bero på att programmet behöver filer som bara
finns på skolan, eller att man av licensskäl inte får köra det hemma.

För att ansluta till datorerna kör man \emph{ssh} (secure shell) mot någon av
loginmaskinerna. Tänk på att loginmaskinerna inte är gjorda för tunga
och långa programkörningar, så använd dem bara till lättare program.

Har man windows hemma kan man ladda hem ett ssh-program som heter
\emph{putty} från \ST{www.chiark.greenend.org.uk/\~{}sgtatham/putty/}.

Starta
programmet. Gå in under \emph{Session} i vänsterspalten och fyll i
namnet på loginmaskinen
under \emph{Host Name (or IP address)}. 
Kontrollera att \emph{SSH} är markerat som
\emph{Protocol}. Klicka
sedan på \emph{Open}. Svara \emph{Yes} i frågerutan och logga sedan in med
ditt namn och lösenord.

Om man kör Linux så räcker det med att man skriver 
\ST{ssh  username@login.student.lth.se} i en terminal för att ansluta.

När du nu har anslutit till skolan kan du starta icke X-baserade program
(program utan grafik) som till exempel egna javaprogram eller matlabprogram
utan grafik. 
Om du däremot vill starta program med grafik (till exempel en firefox, egna 
grafiska java-program eller grafer i Matlab) så behöver du en X-server. 
(Om du försöker starta ett program som kräver X-server så får du 
felmeddelandet \emph{Error: Can't open display}.)


\section{X-server}

Att installera och konfigurera en X-server kräver en del kunskaper. En bra
gratis X-server för Windows är \emph{Cygwin/X} som man kan ladda ned på 
\ST{http://x.cygwin.com/}.

Starta X-servern. Därefter startar du putty, går in under fliken
\emph{Tunnels} och klickar för \emph{Enable X11 forwarding.} Lämna rutan
\emph{location} tom och anslut sedan på vanligt sätt till en logindator.

Ett annat gratisalternativ är \emph{Xming} som finns på
\ST{http://www.straightrunning.com/XmingNotes/}.

%TODO wayland?
Kör du någon Linux hemma är det mycket enklare - då har du nämligen redan
en X-server. I så fall är det bara att logga in som vanligt med ssh men lägga
till \ST{-X} som argument, alltså \ST{ssh -X username@login.student.lth.se}.

\section{Program hemma}

Det finns en del program som kan laddas hem gratis från skolans servrar. Dels
från \ST{program.ddg.lth.se}, dels från Studentportalen 
(\ST{www.student.lu.se}). 

\section{Hjälp och kontakter}

Snabbaste hjälpen får du troligen genom att prata med andra studenter. Om du
inte hittar någon att fråga så kan du göra det via LTH:s studentforum som
finns på \ST{forum.student.lth.se}. Där an man diskutera datorer, studier och
annat.

\subsection{Nyheter}
Nyheter om datorerna hittar man på DDG:s driftinfo på
\ST{http://www.ddg.lth.se/}. När det är nätverksproblem eller annat som ligger
utanför LTH kan man även titta på \ST{http://www.ldc.lu.se/drift}. (LDC är en
organisation inom universitetet som bland annat sköter nätverken.)

\subsection{LTH:s studentsidor}

På \ST{http://www.student.lth.se/} kommer du åt en massa information, bland
annat schema, kursplaner och programhemsidor.

\section{Kursanmälan}

Inför varje läsperiod måste du anmäla dig till de kurser du vill läsa. Detta
gäller dock inte första terminen, men under senare delen av hösten ska du
anmäla dig till kurser på vårterminen. Kursanmälan görs i
kursanmälningssystemet som finns i Studentportalen
(\ST{http://www.student.lu.se/}).


\section{Andra tips}

\begin{itemize}
\item Kopiera filer till en annan användare genom att skriva \ST{scp fil
username@localhost:} och ange den andra användarens lösenord. Glöm inte
kolon på slutet. Filen hamnar i användarens hemkatalog.
\item Kolla kvarvarande diskutrymme med kommandot \emph{quota}. För att hitta
vilka filer eller kataloger som tar upp plats, kör \ST{du -s .??* * | sort
-n}
\item Om du startar ett grafiskt program utan att skriva \& sist på
kommandoraden kan du trycka Ctrl-Z för att tillfälligt stoppa programmet och
sedan \ST{bg} för att fortsätta köra det i bakgrunden. Kommandot \ST{fg}
plockar fram programmet och kör det i förgrunden igen (som om man inte hade
skrivit \&).
\end{itemize}
%%NOTE: Kolla adresserna
%%CHANGE: Ändrade adresserna
\section{Sektionerna}

Här är länkar till de olika sektionerna:

\begin{tabular}{ll}
F/Pi/Nano & \ST{https://www.fsektionen.se/} \\
E & \ST{https://eee.esek.se/} \\
M & \ST{https://www.maskinsektionen.com/} \\
V & \ST{http://www.v-sek.se/} \\
A & \ST{http://www.asektionen.se/} \\
K & \ST{http://www.ksek.se/} \\
D/C & \ST{http://www.dsek.se/} \\
Dokt & \ST{http://www.dokt.tlth.se/} \\
Ing & \ST{http://www.ingsekt.se/} \\
W & \ST{http://www.wsek.se/} \\
I & \ST{http://www.isek.se/} \\
\end{tabular}
\clearpage

%=============================================================================
\chapter{Övningsuppgifter}
%=============================================================================
\label{cha:uppgifter}
\begin{exercise}
  Ta reda på vad din övningsledare heter och anteckna det nedan.
	\vspace{5mm}
	Min övningsledare heter \dotfill
\end{exercise}

\begin{exercise}
	Logga in på datorn med dina kontouppgifter. Om ditt konto inte fungerar kan din övningsledare hjälpa dig lösa problemet.
\end{exercise}

\begin{exercise}
	Öpnna ett terminalfönster genom att trycka på \emindex{Ctrl+Alt+T} eller
	genom att trycka på ikonen längst upp i panelen, sök på \emindex{Terminal} och välj terminal.	

Öppna sedan en editor (\emph{ikonen längst upp i panelen->Text Editor}).
Prova att skriva lite text och kopiera den inom ett fönster och mellan
fönstren.
\end{exercise}

\begin{exercise}
  Prova att skriva text med cat. Lär dig hur man
  skriver \Ctrl{d} för att avsluta inmatningen.
\end{exercise}

\begin{exercise}
  Skapa en fil som med samma namn som ditt förnamn och mata in ditt
  namn och adress. Använd gärna aao i stället för åäö om du har svenska
  bokstäver i ditt förnamn, och undvik accenter.
  \begin{answer}
    Använd \emph{cat > namn.txt} för att skapa och skriva till filen
  \end{answer}
\end{exercise}

\begin{exercise}
  Skriv ut din fil på skärmen och kontrollera att din adress stämmer.
  \begin{answer}
    Använd \emph{cat filnamn} för att läsa från filen
  \end{answer}
\end{exercise}


\begin{exercise}
  Hur gör man för att söka efter en sträng i less? (Använd den inbyggda
  hjälpen.)
  \begin{answer}
    Man skriver \ST{/sträng} för att söka efter \emph{sträng}.
  \end{answer}
\end{exercise}

\begin{exercise}
  Prova att byta namn på din fil och skapa ett par kopior av den. Ta
  sedan bort alla kopior så att det bara finns en fil kvar. Under
  tiden kan du använda ls för att se vilka filer som finns och cat för
  att se deras innehåll.
  \begin{answer}
    använd \emph{mv} för att döpa om (flytta) din fil, och \emph{cp} för att kopiera den
  \end{answer}
\end{exercise}

\begin{exercise}
  Skapa en ny katalog som heter temp. Skapa en kopia av din adressfil
  i den nya katalogen. Försök sedan att ta bort katalogen med
  rmdir. Vad händer?
  \begin{answer}
    rmdir svarar: \ST{rmdir: temp: Directory not empty}. Det går
    alltså inte att ta bort en katalog som innehåller filer. För att
    kunna ta bort katalogen måste man skriva \ST{rm temp/fil} och
    sedan köra rmdir.

    Ett annat sätt är att skriva \ST{rm -r temp}. Det betyder: ta
    bort katalogen temp och \emph{alla} filer och underkataloger som
    finns i temp (\ST{-r} betyder rekursiv). Detta är lite
    farligt om man råkar skriva fel.
  \end{answer}
\end{exercise}
\begin{exercise}
  Skapa en underkatalog som heter \emph{slask}. Prova att kopiera och
  flytta din fil till den nya katalogen, dels under samma namn, dels
  med nytt namn. Döp sedan om katalogen till \emph{adresser} och ta
  bort några filer.  Använd under tiden ls för att se innehållet i
  katalogen.
\end{exercise}

\begin{exercise}
  Vilken katalog pekar \ST{\~{}/./././././adresser/././..} på? Kan
  ovanstående sökväg förkortas och i så fall hur och varför?
  \begin{answer}
    Det är samma sak som din hemmakatalog. Eftersom en ensam punkt
    betyder att man går till den katalog där man redan befinner sig
    kan alla \ST{./} tas bort. \ST{adresser/..} betyder att
    man går ner i \ST{adresser} och sedan upp igen. Alltå kan även
    detta tas bort. Hela sökvägen kan förkortas till \ST{\~{}}.
  \end{answer}
\end{exercise}
\begin{exercise}
  Eftersom en ensam punkt pekar på aktuell katalog kan man inte ha den
  till mycket, eller\ldots? Försök komma på ett tillfälle när den behövs.
  \begin{answer}
    Om man vill kopiera eller flytta en fil från en annan katalog
    \emph{till} aktuell katalog utan att döpa om den skriver man t~ex:
    \begin{ttquote}
      cp adresser/nisse .
    \end{ttquote}
  \end{answer}
\end{exercise}

\begin{exercise}
  Vilka filer kommer att tas bort om man skriver \ST{rm *pr*} om vi har
  filerna från kapitlet \ref{wildcards} Wildcards?
  \begin{answer}
    inpris.dat profile.txt prog prog1.o prog1.p prog2.o prog2.p
    utpris.dat
  \end{answer}
\end{exercise}

\begin{exercise}
  Ta fram manualen för \emph{rm}. Vad betyder det om man skriver
  \ST{rm~-i} i stället för att bara skriva \ST{rm}?
  \begin{answer}
    Det får rm att fråga om tillstånd innan den tar bort några filer.
  \end{answer}
\end{exercise}

\begin{exercise}
  Använd \emph{ls -a \ST{\~}/} för att se dina filer i din hemkatalog.
  Hur skiljer sig \emph{ls -a} mot \emph{ls}?
	\begin{answer}
		\emph{ls -a} visar alla filer som finns i mappen, \emph{ls} visar
		inte filer som börjar med . Punktfiler är ofta konfigurationsfiler
		för olika program och behöver därför inte alltid visas. Man ser också mapparna
		\emph{.} och \emph{..} som pekar på den nuvarande mappen respektive mappen som den nuvarande
		mappen ligger i.
	\end{answer}
\end{exercise}

\begin{exercise}
  Kör programmet \emph{xcalc} genom att skriva \ST{xcalc} i
  ditt terminalfönster och trycka \emph{Return}. Vad händer?
  \begin{answer}
    Det dyker upp en kalkylator. \emph{xcalc} är namnet på
    programmet för en kalkylator.
  \end{answer}
\end{exercise}
\begin{exercise}
  Avsluta kalkylatorn och starta den sedan
  igen genom att skriva \ST{xcalc \&}. Kontrollera att du
  nu kan skriva nya kommandon i terminalfönstret.
\end{exercise}

\begin{exercise}
  Skriv \ST{echo \$PATH} för att ta reda på vilka kataloger som finns i
  din sökväg för program. \emindex{echo} är ett program som helt
  enkelt skriver ut sina argument, prova till exempel med
  \ST{echo~hej} eller \ST{echo *}
\end{exercise}

\begin{exercise}
  Räkna antalet filer i din hemmakatalog och i katalogen \emph{adresser}.
  \begin{answer}
    Du kan använda \emph{ls | grep wc} för att räkna filer i aktuell katalog
  \end{answer}
\end{exercise}
\begin{exercise}
  Vad betyder utmatningen från följande kommando: 
  \mbox{\ST{wc fil $\mid$ wc}}?
  \begin{answer}
    Först räknar man antalet ord i \emph{fil}. Därefter räknar man
    antalet ord i utmatningen från wc.
  \end{answer}
\end{exercise}

\begin{exercise}
  Om ni vill testa webmailen får ni gärna skicka ett brev till
  författaren av denna handledning. Adressen är
  \ST{dat15has@student.lu.se}. Skriv lite synpunkter på handledningen så att
  den kan bli bättre i framtiden. Du kommer att få ett svar på ditt
  brev, men det kan ta lite tid eftersom det kan bli många brev under
  en kort tid.
\end{exercise}


\appendix
\answers
\label{answers}

\onecolumn
\printindex

\cleardoublepage
\thispagestyle{empty}
\hbox{}
\newpage
\onecolumn
\thispagestyle{empty}
\hbox{}\vfill\noindent\small
\centerline{LTH, Tryckeriet i E-huset, Lund \number\year{}}

\end{document}
